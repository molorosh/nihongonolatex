% run as XeLaTeX
%!TEX TS-program = xelatex
\documentclass[44pt, oneside]{article}   	% use "amsart" instead of "article" for AMSLaTeX format
\usepackage{geometry}                		% See geometry.pdf to learn the layout options. There are lots.
\geometry{a4paper, margin=0.8in}                   		% ... or a4paper or a5paper or letterpaper or... 
%\geometry{landscape}                		% Activate for rotated page geometry
\usepackage[parfill]{parskip}    		% Activate to begin paragraphs with an empty line rather than an indent
\usepackage{graphicx}				% Use pdf, png, jpg, or eps§ with pdflatex; use eps in DVI mode
								% TeX will automatically convert eps --> pdf in pdflatex	

\usepackage{hyperref}				% enabled hyperlink support	
%\usepackage{amssymb}

\hypersetup{
    colorlinks=true,
    linkcolor=blue,
    filecolor=magenta,      
    urlcolor=cyan,
    pdftitle={Overleaf Example},
    pdfpagemode=FullScreen,
    }
    
% filename: kanji_commands.tex

\usepackage{xeCJK}
\setCJKmainfont{Hiragino Mincho Pro}
\usepackage{tikz}
% characters with big height and depth
% to give different boxes the same vertical size
\newcommand{\vsizecorrectorhira}{\vphantom{もりぼゃ}}% epkyouka, HanaMinA
\newcommand{\vsizecorrectorkanji}{\vphantom{$\vert$}}

% commands for setting furigana above kanji
\newcommand{\fg}[2]{% #1: kanji, #2: furigana
    %\unskip
    \begin{tikzpicture}[baseline=(kanji.base)]
        \node(kanji)[
            inner sep=0,
        ] {
            \vsizecorrectorkanji%
            #1%
        };
        \node(furigana)[
            above of=kanji,
            node distance=1em,
            inner sep=0,
        ] {%
            \tiny%
            \vsizecorrectorhira%
            #2%
        };
    \end{tikzpicture}%
    %\ignorespaces
}


% commands for setting furigana BELOW kanji
\newcommand{\fgB}[2]{% #1: kanji, #2: furigana
    %\unskip
    \begin{tikzpicture}[baseline=(kanji.base)]
        \node(kanji)[
            inner sep=0,
        ] {
            \vsizecorrectorkanji%
            #1%
        };
        \node(furigana)[
            below of=kanji,
            node distance=1em-2pt,
            inner sep=0,
        ] {%
            \tiny%
            \vsizecorrectorhira%
            #2%
        };
    \end{tikzpicture}%
    %\ignorespaces
}

\newcommand{\fgAB}[3]{% #1: kanji, #2: furigana above, #3: furigana below
    %\unskip
    \begin{tikzpicture}[baseline=(kanji.base)]
        \node(kanji)[
            inner sep=0,
        ] {
            \vsizecorrectorkanji%
            #1%
        };
        \node(furigana-above)[
            above of=kanji,
            node distance=1em,
            inner sep=0,
        ] {%
            \tiny%
            \vsizecorrectorhira%
            #2%
        };
        \node(furigana-below)[
            below of=kanji,
            node distance=1em-2pt,
            inner sep=0,
        ] {%
            \tiny%
            \vsizecorrectorhira%
            #3%
        };
    \end{tikzpicture}%
    %\ignorespaces
}

\newcommand{\?}{
\symbol{"EB08}
}

\newcommand{\fgdate}[4]{% #1: year-num, #2: month-num, #3: day-num, #4: day-fg
#1年#2月\fg{#3日}{#4}
}



\title{\vspace{-2.0cm}Japanese Intermediate 1}
\date{\vspace{-1.0cm}\fgdate{2022}{10}{4}{よっか} }

%%%%%%%%%%%%%%%%%%%%%%%%%%%%%%%%%%%%%%%%%%%%%%
\begin{document}

\maketitle{}

\href{https://padlet.com/coachingteam/2806zrpg7nt2okgi}{PADLETのハイパーリンク}

% we use \protect for fragile (parameterised) commands that may get moved to other places in the document (e.g. section -> table of contents)
\section*{\protect \fg{漢字}{かんじ}や\fg{単語}{たんご}。(kanji and vocab)}

\begin{tabular}{l l l }

\fg{動物}{どうぶつ}(animal) &  \fg{会話}{かわい}(conversation)& \fg{練習}{れんしゅう}(practise/training)\\

\end{tabular}

\section*{\protect \fg{文法}{ぶんぽう}。(grammar)}

\subsection*{\protect Continuing action in the past (again)}

V-ていました。/ V-ていましたか。

\fg{昨日}{きのう}、\fg{公園}{こうえん}でどうしていましたか。(What were you doing in the park yesterday?)

\subsection*{\protect V-\fg{可能系}{かのうけい}とて (potential form + てform)}

V-(られ)て、____ 。/ V-(られ)なくて、____ 。(giving a reason for/cause of one's feelings)

\fg{親}{おや}に\fg{会}{あ}えて、よかったです。(It was good, that I was able to meet my family.)

\fg{友達}{ともだち}に、\fg{会}{あ}えなくて、\fg{残念}{ざんねん}でした。(It was disappointing that I wasn't able to meet my friends.)

\section*{\protect \fg{質問}{しつもん}で\fg{会話}{かいわ}の\fg{燃焼}{れんしゅう}することについては。(conversation question practise)}

Example questions (\textit{with no particular theme}):

\fg{暗殺者}{あんさつしゃ}になることは\fg{如何}{どう}でしたか。(How did you become an assassin?)

これは\fg{誰}{だれ}のナイフですか。(Whose knife is this?)

だれと\fg{刑務所}{けいむしょ}に\fg{行}{い}きますか。(Who are you going to prison with?)

どんなナイフが\fg{一番}{いちばん}\fg{好}{す}きなナイフですか。(what is your favourite type of knife?)

どうしてリトアニアに\fg{行}{い}きましたか。(Why did you go to Lithuania?)

\fg{昨日}{きのう}、\fg{実験所}{じっけんしょ}でどうしていましたか。(What were you doing in the research facility yesterday?)

\fg{警官}{けいかん}はどこにいりますか。(Where are the police officers?)

オリエント\fg{急行}{きゅうこう}は\fg{何時}{なんじ}にイスタンブールに\fg{着}{つ}きますか。(What time does the Orient Express reach Istanbul?)

\fg{李}{すもも}\fg{先生}{せんせい}はいつ\fg{死}{し}にましたか。(When did Professor Plum die?)







\end{document}  