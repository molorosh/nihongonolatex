% run as XeLaTeX
%!TEX TS-program = xelatex
\documentclass[44pt, oneside]{article}   	% use "amsart" instead of "article" for AMSLaTeX format
\usepackage{geometry}                		% See geometry.pdf to learn the layout options. There are lots.
\geometry{a4paper}                   		% ... or a4paper or a5paper or letterpaper or... 
%\geometry{landscape}                		% Activate for rotated page geometry
\usepackage[parfill]{parskip}    		% Activate to begin paragraphs with an empty line rather than an indent
\usepackage{graphicx}				% Use pdf, png, jpg, or eps§ with pdflatex; use eps in DVI mode
								% TeX will automatically convert eps --> pdf in pdflatex	

\usepackage{hyperref}				% enabled hyperlink support	
%\usepackage{amssymb}

\usepackage{xeCJK}
\setCJKmainfont{Hiragino Mincho Pro}
\usepackage{tikz}
% characters with big height and depth
% to give different boxes the same vertical size
\newcommand{\vsizecorrectorhira}{\vphantom{もりぼゃ}}% epkyouka, HanaMinA
\newcommand{\vsizecorrectorkanji}{\vphantom{$\vert$}}

% commands for setting furigana above kanji
\newcommand{\fg}[2]{% #1: kanji, #2: furigana
    %\unskip
    \begin{tikzpicture}[baseline=(kanji.base)]
        \node(kanji)[
            inner sep=0,
        ] {
            \vsizecorrectorkanji%
            #1%
        };
        \node(furigana)[
            above of=kanji,
            node distance=1em,
            inner sep=0,
        ] {%
            \tiny%
            \vsizecorrectorhira%
            #2%
        };
    \end{tikzpicture}%
    %\ignorespaces
}


% commands for setting furigana BELOW kanji
\newcommand{\fgB}[2]{% #1: kanji, #2: furigana
    %\unskip
    \begin{tikzpicture}[baseline=(kanji.base)]
        \node(kanji)[
            inner sep=0,
        ] {
            \vsizecorrectorkanji%
            #1%
        };
        \node(furigana)[
            below of=kanji,
            node distance=1em-2pt,
            inner sep=0,
        ] {%
            \tiny%
            \vsizecorrectorhira%
            #2%
        };
    \end{tikzpicture}%
    %\ignorespaces
}

\newcommand{\fgAB}[3]{% #1: kanji, #2: furigana above, #3: furigana below
    %\unskip
    \begin{tikzpicture}[baseline=(kanji.base)]
        \node(kanji)[
            inner sep=0,
        ] {
            \vsizecorrectorkanji%
            #1%
        };
        \node(furigana-above)[
            above of=kanji,
            node distance=1em,
            inner sep=0,
        ] {%
            \tiny%
            \vsizecorrectorhira%
            #2%
        };
        \node(furigana-below)[
            below of=kanji,
            node distance=1em-2pt,
            inner sep=0,
        ] {%
            \tiny%
            \vsizecorrectorhira%
            #3%
        };
    \end{tikzpicture}%
    %\ignorespaces
}

\newcommand{\?}{
\symbol{"EB08}
}



\title{Brief Article}
\author{The Author}
\date{}	
						% Activate to display a given date or no date
%%%%%%%%%%%%%%%%%%%%%%%%%%%%%%%%%%%%%%%%%%%%%%
\begin{document}
%\maketitle
%\section{}
%\subsection{}

This document uses TikZ for furigana support.  It is based on 
\href{https://tex.stackexchange.com/questions/95729/typesetting-furigana-above-and-below-original-text}{this tex.stackexchange.com question}


sdfgsdfg dfgdg some basic japanese text
私は日本語の勉強します! 
lksjdflkg

⓿

Japanese uses different alphabets, one of those is \fg{漢字}{かんじ} (Kanji).

Japanese uses different alphabets, one of those is \fgB{漢字}{かんじ} (Kanji) and this time the furigana is below the text.

Kanji can have different readings: \fgAB{日}{にち}{ひ}.   [.  \?  ]

Conversation Cues

会話 / かいわ \symbol{"00D7} \symbol{"203B} \symbol{"EB08}

Beyond just talking and listening, a typical conversation has some special purpose phrases.

Phatics - words (or just noises) to indicate that you are you haven't finished speaking yet (even it you have temporarily stopped talking).  In English this is things like: "um..." and "er..." and "like..." ... 

Backchanelling - small responses to indicate you are still listening, but don't intend to take over the conversation yet.  In English this is things like: 'Oh?', "Really?", "uhuh", "uuh?", "wow", 'yeah', 'I see', "no way"

Questions - small sentences to get more detailed information from the speaker (based on what they were saying)

Asking for translation assistance in Japanese.

If you are talking to a Japanese person who is also fluent in English, it is useful to be able to ask a few translation questions like: "How to you say APPLE in Japanese?" or "What does RINGO mean in English".

One of the key words is \fg{意味}{いみ} which means: "meaning" or "linguistic definition"

What does \?{} mean?

\?{}はどういう意味ですか。

\?{} is the meaning.

\?{}はという意味です。



Mental Block: When you just can't think of the word!

%this form of \href uses \nolinkrurl to allow it to handle line-breaks better
\href{mailto:molorosh@gmail.com}{\nolinkurl{molorosh@gmail.com} }


\end{document}  